\documentclass[11p]{article}
% Packages
\usepackage{amsmath}
\usepackage{graphicx}
\usepackage{fancyheadings}
\usepackage[swedish, english]{babel}
\usepackage[
    backend=biber,
    style=authoryear-ibid,
    sorting=ynt
]{biblatex}
\usepackage[utf8]{inputenc}
\usepackage[T1]{fontenc}
\usepackage{titlesec}
\usepackage{hyperref}
%Källor
\addbibresource{references.bib}
\graphicspath{ {./images/} }

% Lite variabler
\def\email{alvin.hogdal@elev.ga.ntig.se}
\def\foottitle{Gymnasiearbete} % Kanske borde ändras
\def\name{Alvin Högdal}

\title{Gymnasiearbete \\ \small Gymnasiearbete}
\author{\name}
\date{\today}


\begin{document}



% fixar sidfot
    \lfoot{\footnotesize{\name \\ \email}}
    \rfoot{\footnotesize{\today}}
    \lhead{\sc\footnotesize\foottitle}
    \rhead{\nouppercase{\sc\footnotesize\leftmark}}
    \pagestyle{fancy}
    \renewcommand{\headrulewidth}{0.2pt}
    \renewcommand{\footrulewidth}{0.2pt}

% i Sverige har vi normalt inget indrag vid nytt stycke
    \setlength{\parindent}{0pt}
% men däremot lite mellanrum
    \setlength{\parskip}{10pt}
    \begin{otherlanguage}{swedish}


        \begin{titlepage}
            \centering

            % Title and subtitle are enclosed between two rules.
            \rule{\textwidth}{1pt}

            % Title
            \vspace{.7\baselineskip}
            {\huge \textbf{Title}}

            % Subtitle
            \vspace*{.5cm}
            {\LARGE Subtitle}

            \rule{\textwidth}{1pt}

            \vspace{1cm}

            % Set this size for the remaining titlepage.
            \large

            % Authors side by side, using two minipages as a trick.
            \begin{minipage}{.5\textwidth}
                \centering
                \name \\
                {\normalsize \url{alvinhogdal@elev.ga.ntig.se}}
            \end{minipage}%

            \vspace{3cm}

            % Report logo.
            \includegraphics[width=0.4\textwidth]{../images/NTI Gymnasiet_Symbol_print_svart.png}

            \vfill

            % University and date information at the bottom of the titlepage.
            NTI Gymnasiet Umeå \\
            Teknikprogrammet\\
            Gymnasiearbete\\
            Datum: \today \\
            Handledare: Jens Andreasson
        \end{titlepage}
        %\maketitle
        %\begin{center}
        %\includegraphics[width=0.4\textwidth]{../images/NTI Gymnasiet_Symbol_print_svart.png}
        %\end{center}
        
        
    \end{otherlanguage}


    \begin{otherlanguage}{swedish}
    \tableofcontents

    \newpage

% i Sverige har vi normalt inget indrag vid nytt stycke\setlength{\parindent}{0pt}
% men däremot lite mellanrum
\setlength{\parskip}{10pt}

\section{Inledning}

\subsection{Syfte & Frågeställning}
    Undersöka hur priset på liknande datorspel har ändrats över tid.

\section{Bakgrund}

    \subsection{datorspel}
    Datorspel är
    Enligt \textcite{ComputerSweden} så defineras datorspel som något spel som kräver en dator.
    \subsection{utvecklare}
    Ett datorspel kräver att någon skapar det.
    Dessa personer eller företag kallas utvecklare.
    Hur många utvecklare ett datorspel har kan variera kraftigt beroende på hur stort spelet är.
    Det finns ingen gräns på hur många personer som kan utveckla ett spel men det krävs minst en.
    \subsection{utgivare}

    \subsection{Shooter}
    test
    \subsubsection{First person shooter}
    test
    \subsection{Downloadable content}
    test
    \subsection{Battle pass}
    test
    \subsection{Micro-transactions}
    test
\section{Metod}

     Undersökningen går ut på att hitta information på när de valda spelen släpptes.
     Hur mycket det kostade då. %kanske utväcklingstid
     Vem som utvecklade spelet och vem som är utgivare.
     Alla dlc och vad de kostade när de släpptes.

    \printbibliography

    \end{otherlanguage}

\end{document}
