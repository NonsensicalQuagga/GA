\documentclass[11p]{article}
% Packages
\usepackage{amsmath}
\usepackage{graphicx}
\usepackage{fancyheadings}
\usepackage[swedish]{babel}
\usepackage[
    backend=biber,
    style=authoryear-ibid,
    sorting=ynt
]{biblatex}
\usepackage[utf8]{inputenc}
\usepackage[T1]{fontenc}
%Källor
\addbibresource{references.bib}
\graphicspath{ {./images/} }

% Lite variabler
\def\email{alvin.hogdal@ga.ntig.se}
\def\foottitle{PMmall}
\def\name{Alvin Högdal}

\title{PMmall \\ \small Gymnasiearbete}
\author{\name}
\date{\today}

\begin{document}

% fixar sidfot
\lfoot{\footnotesize{\name \\ \email}}
\rfoot{\footnotesize{\today}}
\lhead{\sc\footnotesize\foottitle}
\rhead{\nouppercase{\sc\footnotesize\leftmark}}
\pagestyle{fancy}
\renewcommand{\headrulewidth}{0.2pt}
\renewcommand{\footrulewidth}{0.2pt}

% i Sverige har vi normalt inget indrag vid nytt stycke
\setlength{\parindent}{0pt}
% men däremot lite mellanrum
\setlength{\parskip}{10pt}

\maketitle

\section{Bakgrund}
Web Content Accessibility Guidelines eller WCAG är rekomendationer om hur en hemsida borde se ut eller fungera på webben.
Om man följer WCAG riktlinjerna så är målet att personer med olika funktionsnedsättningar som exempelvis blindhet, dövhet, ljudkänslighet och ... med mera.
Att följa dessa riktlinjer gör oftast så att hemsidan är mer användbar generellt.\parencite{WCAG}


En skärmläsare är en mjukvara som omvandlar textinnehåll på en webbsida till ljud eller punktskrift.
Den omvandlar också alternativ text så att en användare kan ta del av all information på hemsidan.\parencite{audioeye}


Alternativ text är någonting som personen som har gjort hemsidan kan lägga till på grafiskt innehåll på hemsidan.
Enligt WCAG 2.0 1.1.1 så ska allt innehåll som inte är text ha ett text alternativ som uppfyller motsvarande syfte med vissa undantag.
Exempelvis så behövs inte alternativtext om innehållet är dekorativt.


Ide: Kolla hur 2 olika skärmläsare fungerar på en hemsida (kommer förmodligen ha fler än 2)

Varför tillgänglighet?

Information om blinda och dyslektiska.

Hitta 2 olika skärmläsare och berätta om dom.




\printbibliography

\end{document}
