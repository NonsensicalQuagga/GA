\documentclass[11p]{article}
% Packages
\usepackage{amsmath}
\usepackage{graphicx}
\usepackage{fancyheadings}
\usepackage[swedish, english]{babel}
\usepackage[
    backend=biber,
    style=authoryear-ibid,
    sorting=ynt
]{biblatex}
\usepackage[utf8]{inputenc}
\usepackage[T1]{fontenc}
\usepackage{titlesec}
\usepackage{hyperref}
%Källor
\addbibresource{references.bib}
\graphicspath{ {./images/} }

% Lite variabler
\def\email{alvin.hogdal@elev.ga.ntig.se}
\def\foottitle{Gymnasiearbete} % Kanske borde ändras
\def\name{Alvin Högdal}

\title{Gymnasiearbete \\ \small Gymnasiearbete}
\author{\name}
\date{\today}


\begin{document}



% fixar sidfot
    \lfoot{\footnotesize{\name \\ \email}}
    \rfoot{\footnotesize{\today}}
    \lhead{\sc\footnotesize\foottitle}
    \rhead{\nouppercase{\sc\footnotesize\leftmark}}
    \pagestyle{fancy}
    \renewcommand{\headrulewidth}{0.2pt}
    \renewcommand{\footrulewidth}{0.2pt}

% i Sverige har vi normalt inget indrag vid nytt stycke
    \setlength{\parindent}{0pt}
% men däremot lite mellanrum
    \setlength{\parskip}{10pt}
    \begin{otherlanguage}{swedish}


        \begin{titlepage}
            \centering

            % Title and subtitle are enclosed between two rules.
            \rule{\textwidth}{1pt}

            % Title
            \vspace{.7\baselineskip}
            {\huge \textbf{Title}}

            % Subtitle
            \vspace*{.5cm}
            {\LARGE Subtitle}

            \rule{\textwidth}{1pt}

            \vspace{1cm}

            % Set this size for the remaining titlepage.
            \large

            % Authors side by side, using two minipages as a trick.
            \begin{minipage}{.5\textwidth}
                \centering
                \name \\
                {\normalsize \url{alvinhogdal@elev.ga.ntig.se}}
            \end{minipage}%

            \vspace{3cm}

            % Report logo.
            \includegraphics[width=0.4\textwidth]{../images/NTI Gymnasiet_Symbol_print_svart.png}

            \vfill

            % University and date information at the bottom of the titlepage.
            NTI Gymnasiet Umeå \\
            Teknikprogrammet\\
            Gymnasiearbete\\
            Datum: \today \\
            Handledare: Jens Andreasson
        \end{titlepage}
        %\maketitle
        %\begin{center}
        %\includegraphics[width=0.4\textwidth]{../images/NTI Gymnasiet_Symbol_print_svart.png}
        %\end{center}
        
        
    \end{otherlanguage}


    \begin{otherlanguage}{swedish}
    \tableofcontents

    \newpage

% i Sverige har vi normalt inget indrag vid nytt stycke\setlength{\parindent}{0pt}
% men däremot lite mellanrum
\setlength{\parskip}{10pt}

\section{Inledning}
    Datorspel är fantastiska.%subjekvit


\subsection{Syfte & Frågeställning}
    Undersöka hur priset på liknande datorspel har ändrats över tid.


    Jämföra hur prisutvecklingen har varit på två liknande spel
\section{Bakgrund}

    \subsection{datorspel}
    Datorspel defineras av \textcite{ComputerSweden} som ett spel som kräver någon form av en dator.
    Detta inkluderar fler saker än bara spel på persondatorer.
    Det inkluderar bland annat spelkonsoler, arkadspel och

    \subsection{utvecklare}
    Ett datorspel kräver att någon skapar det.
    Dessa personer eller företag som skapar ett spel har samlingsnamnet utvecklare.
    Alla som har varit med och skapat spelet ingår i namnet utvecklare.
    Hur många utvecklare ett datorspel har kan variera kraftigt beroende på hur stort spelet är.
    Det finns ingen gräns på hur många personer som kan utveckla ett spel men det krävs minst en.

    \subsection{utgivare}
    Utgivare är personerna eller företagen som ansvarar för att marknadsföra spelet.
    Det kan vara exakt samma grupp eller företag som skapat spelet.
    Om utvecklaren har presenterat en spel ide för investerare så är det oftast företaget som investerade i iden som är utgivare.


    \subsection{Shooter}
    En \("\)shooter\("\) är en genre på datorspel.
    Att ett spel är en \("\)shooter\("\) innebär att en av huvudmekanikerna i spelet är att skjuta någon projektil på någon figur.
    Detta kan vara allt från att skjuta pilar på måltavlor till att skjuta kulor på personer.

    \subsubsection{First person shooter}
    En \("\)First person shooter\("\) eller FPS är en \("\)shooter\("\) som utgår från den spelbara karaktärens perspektiv.
    Ett exempel på en FPS är datorspelet \("\)Counter-Strike 2\("\) som är utvecklat och marknadsfört av \textcite{CounterStrike}

    \subsection{Downloadable content}
    Downloadable content eller DLC är innehåll till ett datorspel som utvecklaren och utgivaren av någon anledning vill hålla separat.
    Det kan vara en uppdatering som ändrar någonting, resurser som används i spelet, kosmetiska ändringar, nya banor eller vapen\ldots
    En DLC kan vara vad som helst.
    Det finns ingen gräns för hur mycket DLC ett datorspel kan ha och det finns inte häller någon gräns för hur mycket det kan kosta.
    Generellt sätt så är priset för DLC lägre än vad själva spelet

    \subsection{Battle pass}
    Battle pass är ett system där du köper rätten till att få något innehåll i ett datorspel.
    Med ett Battle pass så får du innehållet genom att spela spelet.
    Medans du spelar spelet så låser du upp nya \("\)tiers\("\).
    En tier är en låda med något innehåll.
    Innan man köper ett battle pass kan man oftast se hur många tiers det har och vad som finns i dom.
    Generellt sätt så är battle pass tidsbegränsade och man låser upp tiers utan att köpa det men man får inte innehållet tills man köper det.
    Oftast så kan man också betala för att låsa upp nästa tier efter man köpt battle passet.

    \subsection{Micro-transactions}
    test

    \subsubsection{Lootbox}
    Lootboxes är en form av micro-transaction där du köper en låda som ger dig någonting.
    Alltså du köper någonting fast du vet inte vad det är.
    I en lootbox så fins det oftast väldigt många olika saker med olika chans att få dom.
    Sakerna som en spelare vill ha har oftast väldigt låg chans.
    Låt os säga att du vill ha ett specifikt \("\)Legendary item\("\).
    Det fins en 1\% chans att få en legendarisk sak och det finns 5 olika.
    Alltså en \(1/500\) eller 0,2\% chas att få det du vill.
    Detta system liknar spelmaskiner som finns på kasinon med den största skillnaden att du får pengar på kasinon och saker inuti spelet från lootboxes.

    \subsubsection{inflation}




\section{Metod}

     Undersökningen går ut på att hitta information på när de valda spelen släpptes.
     Hur mycket det kostade då. %kanske utväcklingstid
     Vem som utvecklade spelet och vem som är utgivare.
     Alla dlc och vad de kostade när de släpptes.

    \printbibliography[title={Referenser}]

    \end{otherlanguage}

\end{document}
